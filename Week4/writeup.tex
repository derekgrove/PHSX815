\documentclass[11pt]{article}
\usepackage[top=1in, bottom=1in, left=1in, right=1in]{geometry}
\usepackage{hyperref}
\title{PHSX815 Project 1}
\author{Derek Grove}

\begin{document}
\maketitle
\subsection*{Introduction}
For this project I will be simulating an electron-positron collision at a specified collision energy. You will be able to configure what collision energy you want and then the program will calculate which final-state products are allowed based on their rest masses. This is a toy model, this interaction is simplified by ignoring any higher order contributions or QFT nuances. Since the initial state is neatural (total charge = 0) the final states must also have zero net charge, so our possible products are all six leptons paired with their anti particle: $\nu_e\bar{\nu_e}$, $\nu_\mu\bar{\nu_\mu}$, $\nu_\tau\bar{\nu_\tau}$, $ee^+$, $\mu\bar{\mu}$, $\tau\bar{\tau}$, and then all 6 quarks paired with their anti particle: $u\bar{u}$, $d\bar{d}$, $c\bar{c}$, $s\bar{s}$, $t\bar{t}$, $b\bar{b}$.
\end{document}